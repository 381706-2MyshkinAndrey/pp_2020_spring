\documentclass{report}

\usepackage[english, russian]{babel}
\usepackage{geometry}
\usepackage{listings}
\usepackage[T2A]{fontenc}
\usepackage[14pt]{extsizes}
\usepackage{color}
\usepackage{graphicx}
\usepackage[titles]{tocloft}
\usepackage[hyphens]{url}

\geometry{a4paper,top=2cm,bottom=2cm,left=2cm,right=1.5cm}
\setlength{\parskip}{0.5cm}
\setcounter{tocdepth}{4}
\setcounter{secnumdepth}{4}
\sloppy

\definecolor{darkgreen}{rgb}{0.0, 0.5, 0.0}
\lstset{
    language=C++,
    basicstyle=\footnotesize,
    numbers=left,
    numberstyle=\tiny,
    numberfirstline=true,
    numbersep=5pt,
    keywordstyle=\color{blue}\bfseries,
    commentstyle=\color{darkgreen},
    stringstyle=\color{red},
    showspaces=false,
    showstringspaces=false,
    captionpos=t,
    breaklines=true,
    breakatwhitespace=false,
    extendedchars=true,
    frame=tb,
    title=\lstname,
}

\begin{document}
    \begin{titlepage}
        \begin{center}
            Федеральное государственное автономное образовательное учреждение высшего образования \\
            "Национальный исследовательский Нижегородский государственный университет им. Н.И. Лобачевского" (ННГУ) \\
            Институт информационных технологий, математики и механики

            \vspace{\fill}

            \textbf{\largeОтчет по лабораторной работе}
            \textbf{\large«Вычисление многомерных интегралов с использованием многошаговой схемы (метод трапеций)»}

            \vspace{\fill}

            \hfill\parbox{8cm}{
                \hspace*{5cm}\hspace*{-5cm}\textbf{Выполнил:} \\ студент группы 381708-1 \\ Астафьева Ирина Алексеевна \\ \\ \\
                \hspace*{5cm}\hspace*{-5cm}\textbf{Проверил:}\\ доцент кафедры МОСТ, \\ кандидат технических наук \\ Сысоев Александр Владимирович
                \hbadness=20000
            }

            \vspace{\fill}

            Нижний Новгород \\ 2020 г.
        \end{center}
    \end{titlepage}
    
    \setcounter{page}{2}
    \setlength{\cftsecindent}{0em}
    \setlength{\cftsubsecindent}{1.25em}
    \setlength{\cftsubsubsecindent}{2.5em}
    \setlength{\cftsubsubsecnumwidth}{1.25em}
    \tableofcontents
    
    \newpage
    
    \section*{Введение}
    \addcontentsline{toc}{section}{Введение}
    \par Метод трапеций — метод численного интегрирования функции \textit{n} переменных, заключающийся в замене на каждом элементарном отрезке подынтегральной функции на многочлен \textit{n}-ой степени. Площадь под графиком функции аппроксимируется прямоугольными трапециями.
    \par Разделение пределов интегрирования на элементарные отрезки, для каждого из которых интеграл высчитывается независимо, открывает возможности ускорения работы алгоритма путем распараллеливания вычислений интеграла на элементарных отрезках.
    
    \newpage
    
    \section*{Постановка задачи}
    \addcontentsline{toc}{section}{Постановка задачи}
    \par В ходе данной лабораторной работы неообходимо:
    \begin{itemize}
        \item построить алгоритм вычисления многомерных интегралов методом трапеций;
        \item написать последовательную реализацию алгоритма;
        \item написать параллельную реализацию алгоритма, используя OpenMP и TBB;
        \item проверить корректность работы алгоритма;
        \item провести сравнения времени выполнения вычислений и сделать выводы об эффективности реализаций.
    \end{itemize}
    \newpage
    
    \section*{Описание алгоритма}
    \addcontentsline{toc}{section}{Описание алгоритма}
    \par Для вычисления многомерного интеграла методом трапеций использован алгоритм, схожий с алгоритмом для вычисления одомерного интеграла (\ref{eq:normal}).
    \begin{equation}
        \label{eq:normal}
        \sum^b_a\frac{f(t_i) + f(t_{i-1})}{2}\left(t_i-t_{i-1}\right)
    \end{equation}
    \par На вход алгоритм получает функцию, которую необходимо проинтегрировать, пределы интегрирования для каждой переменной функции и параметры разделения, указывающие, на какое количество отрезков должен быть разделен предел интегрирования каждой переменной.
    \par При каждой итерации вычисляется два набора параметров, различающихся лишь одним параметром на $\delta$. То есть пара наборов параметров для функции трех переменных может выглядеть следующим образом:
    \begin{itemize}
        \item ($x_i, y_j, z_k$) и ($x_i, y_j, z_{k-1}$);
        \item ($x_i, y_j, z_k$) и ($x_i, y_{j-1}, z_k$);
        \item ($x_i, y_j, z_k$) и ($x_{i-1}, y_j, z_k$).
    \end{itemize}
    \par Среднее арифметическое значений функции для этих двух наборов параметров прибавляется к общему результату. После вычисления средних арифметических для всех возможных пар параметров, результат умножается на $\delta$ по каждой переменной.
    \newpage
    
    \section*{Описание схемы распараллеливания}
    \addcontentsline{toc}{section}{Описание схемы распараллеливания}
    \par Так как каждая итерация независима от любой другой, необходимо разделить число итераций между потоками. Число итераций всегда равно произведению параметров разделения, поступивших на вход алгоритма.
    \par Допустим, что на вход алгоритм получил функцию трех переменных и параметры разделения - (10, 10, 10). Алгоритм выполняется 4 потоками. Тогда схема разделения итераций между потоками представлена в таблице 1. Если это же задание выполняется 3 потоками, то схема распределения будет выглядеть как в таблице 2.
    \begin{table}[htbp]
        \centering
        \begin{tabular}{|c|c|}
            \hline
            Поток & Диапазон  \\ \hline
            0     & 0 - 249   \\ \hline
            1     & 250 - 499 \\ \hline
            2     & 500 - 749 \\ \hline
            3     & 750 - 999 \\ \hline
        \end{tabular}
        \caption{Диапазоны итераций для 4 потоков}
        \label{tab:example}
    \end{table}
    \begin{table}[htbp]
        \centering
        \begin{tabular}{|c|c|}
            \hline
            Поток & Диапазон  \\ \hline
            0     & 0 - 333   \\ \hline
            1     & 334 - 666 \\ \hline
            2     & 667 - 999 \\ \hline
        \end{tabular}
        \caption{Диапазоны итераций для 3 потоков}
        \label{tab:example}
    \end{table}
    \newpage
    
    \section*{Описание программной реализации}
    \addcontentsline{toc}{section}{Описание программной реализации}
    \par В этом разделе представлены краткие описания параллельных реализации. Полный код можно посмотреть в Приложении.
    
    \subsection*{Реализация с использованием OpenMP}
    \addcontentsline{toc}{subsection}{Реализация с использованием OpenMP}
    \par Используя функции \verb|omp\_get\_num\_threads()| и \verb|omp\_get\_thread\_num()| определяются количество потоков и номер текущего потока и с помощью этих значений вычисляются итерации, которые должен выполнить текущий поток.
    
    \subsection*{Реализация с использованием TBB}
    \addcontentsline{toc}{subsection}{Реализация с использованием TBB}
    \par Результат вычисляется с помощью \verb|tbb:parallel_reduce|. Эта функция автоматически разделяет итерации между потоками, а затем собирает результаты со всех потоков и производит над ними конкретную операцию, в данном случае - сложение.
    \newpage
    
    \section*{Сравнение времени выполнения версий}
    \addcontentsline{toc}{section}{Сравнение времени выполнения версий}
    \par Конфигурация системы:
    \begin{itemize}
        \item Процессор: Intel(R) Core(TM) i7-8550U CPU @ 1.80GHz 1.99GHz
        \item Оперативная память: 20 GB
        \item Операционная система: Windows 10 Pro
    \end{itemize}
    
    \par Формулировка задания:
    \begin{itemize}
        \item Функция: $log_{10}(2x^2) + \sqrt{z} + 5y$
        \item Пределы интегрирования: \{(1, 2), (-13, 5), (3, 7)\}
        \item Параметры разделения: (500, 500, 500)
    \end{itemize}
    
    \begin{table}[htbp]
        \centering
        \begin{tabular}{|c|c|c|c|}
            \hline
                             & 1       & 2       & 4        \\ \hline
            Последовательная & \multicolumn{3}{c|}{15.7797} \\ \hline
            OpenMP           & 15.6586 & 13.6165 & 9.00504  \\ \hline
            TBB              & 15.6797 & 13.7758 & 7.50737  \\ \hline
        \end{tabular}
        \caption{Результаты эксперимента}
        \label{tab:times}
        По вертикали - число потоков. \\
        По горизонтали - тестируемая реализация. \\
        Время указано в секундах.
    \end{table}
    \newpage
    
    \section*{Проверка корректности вычислений}
    \addcontentsline{toc}{section}{Проверка корректности вычислений}
    \par Погрешность вычислений напрямую зависит от параметров разделения, чем больше параметры разделения тем выше точность. Был проведен эксперимент, который это наглядно показывает.
    
    \par Формулировка задания:
    \begin{itemize}
        \item Функция: $log_{10}(2x^2) + \sqrt{z} + 5y$
        \item Пределы интегрирования: \{(1, 2), (-13, 5), (3, 7)\}
        \item Правильный ответ: -1234.28
    \end{itemize}
    
    \begin{table}[htbp]
        \centering
        \begin{tabular}{|c|c|c|c|c|c|}
            \hline
            Параметры разделения & 10       & 50       & 100      & 500      & 1000     \\ \hline
            Последовательная     & -1206.74 & -1229.86 & -1232.14 & -1233.86 & -1234.07 \\ \hline
            OpenMP               & -1206.74 & -1229.86 & -1232.14 & -1233.86 & -1234.07 \\ \hline
            TBB                  & -1206.74 & -1229.86 & -1232.14 & -1233.86 & -1234.07 \\ \hline
        \end{tabular}
        \caption{Результаты эксперимента}
        \label{tab:results}
    
    \end{table}
    
    \par Результаты этого эксперимента также показывают, что при одинаковых входных данных разные версии выдают одинаковый результат, что в свою очередь указывает на корректную реализацию этих версий.
    \par Программно проверка корректности вычислений осуществляется с помощью UNIT-тестов.
    \newpage
    
    \section*{Заключение}
    \addcontentsline{toc}{section}{Заключение}
    \par В процессе данных лабораторных работ были разработаны последовательная и две параллельных реализации алгоритма вычисления многомерных интегралов методом трапеций. Была проведена оценка скорости работы всех трех реализаций и их корректность.
    \par Технологии OpenMP и TBB оказались одинаково эффективными при решении данной задачи.
    \newpage
    
    \section*{Список литературы}
    \addcontentsline{toc}{section}{Список литературы}
    \begin{enumerate}
        \item Сысоев А.В., Мееров И.Б., Сиднев А.А. Средства разработки параллельных программ для систем с общей памятью. Библиотека Intel Threading Building Blocks. Учебно-методические материалы по программе повышения квалификации «Технологии высокопроизводительных вычислений для обеспечения учебного процесса и научных исследований». Нижний Новгород, 2007, 86 с.
        \item Clever Students: Метод трапеций URL: \url{http://www.cleverstudents.ru/integral/method_of_trapezoids.html}
    \end{enumerate}
    \newpage
    
    \section*{Приложение}
    \addcontentsline{toc}{section}{Приложение}
    \subsection*{Исходный код}
    \addcontentsline{toc}{subsection}{Исходный код}

    \subsubsection*{Лабораторная работа №1. Последовательная версия}
    \addcontentsline{toc}{subsubsection}{Лабораторная работа №1. Последовательная версия}
    \lstinputlisting[language=C++, caption=Последовательная версия. Заголовочный файл]{../../../../modules/task_1/astafeva_i_trapezoidal_rule/trapezoidal_rule.h}
    \lstinputlisting[language=C++, caption=Последовательная версия. Файл с реализацией]{../../../../modules/task_1/astafeva_i_trapezoidal_rule/trapezoidal_rule.cpp}
    \lstinputlisting[language=C++, caption=Последовательная версия. Файл с тестами]{../../../../modules/task_1/astafeva_i_trapezoidal_rule/main.cpp}
    \newpage
    
    \subsubsection*{Лабораторная работа №2. Параллельная версия с использованием OpenMP}
    \addcontentsline{toc}{subsubsection}{Лабораторная работа №2. Параллельная версия с использованием OpenMP}
    \lstinputlisting[language=C++, caption=Последовательная версия. Заголовочный файл]{../../../../modules/task_2/astafeva_i_trapezoidal_rule/trapezoidal_rule.h}
    \lstinputlisting[language=C++, caption=Последовательная версия. Файл с реализацией]{../../../../modules/task_2/astafeva_i_trapezoidal_rule/trapezoidal_rule.cpp}
    \lstinputlisting[language=C++, caption=Последовательная версия. Файл с тестами]{../../../../modules/task_2/astafeva_i_trapezoidal_rule/main.cpp}
    \newpage
    
    \subsubsection*{Лабораторная работа №3. Параллельная версия с использованием TBB}
    \addcontentsline{toc}{subsubsection}{Лабораторная работа №3. Параллельная версия с использованием TBB}
    \lstinputlisting[language=C++, caption=Последовательная версия. Заголовочный файл]{../../../../modules/task_3/astafeva_i_trapezoidal_rule/trapezoidal_rule.h}
    \lstinputlisting[language=C++, caption=Последовательная версия. Файл с реализацией]{../../../../modules/task_3/astafeva_i_trapezoidal_rule/trapezoidal_rule.cpp}
    \lstinputlisting[language=C++, caption=Последовательная версия. Файл с тестами]{../../../../modules/task_3/astafeva_i_trapezoidal_rule/main.cpp}
    
\end{document}
