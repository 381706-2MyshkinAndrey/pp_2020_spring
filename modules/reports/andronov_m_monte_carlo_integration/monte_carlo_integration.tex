\documentclass{report}

\usepackage[T2A]{fontenc}
\usepackage[utf8]{luainputenc}
\usepackage[english, russian]{babel}
\usepackage[pdftex]{hyperref}
\usepackage[14pt]{extsizes}
\usepackage{listings}
\usepackage{color}
\usepackage{geometry}
\usepackage{enumitem}
\usepackage{multirow}
\usepackage{graphicx}
\usepackage{indentfirst}
\usepackage{caption}

\geometry{a4paper,top=2cm,bottom=3cm,left=2cm,right=1.5cm}
\setlength{\parskip}{0.5cm}
\setlist{nolistsep, itemsep=0.3cm,parsep=0pt}

\lstset{language=C++,
		basicstyle=\footnotesize,
		keywordstyle=\color{blue}\ttfamily,
		stringstyle=\color{red}\ttfamily,
		commentstyle=\color{green}\ttfamily,
		morecomment=[l][\color{magenta}]{\#}, 
		tabsize=4,
		breaklines=true,
  		breakatwhitespace=true,
  		title=\lstname,       
}

\makeatletter
\renewcommand\@biblabel[1]{#1.\hfil}
\makeatother

\begin{document}

\begin{titlepage}

\begin{center}
Министерство науки и высшего образования Российской Федерации
\end{center}

\begin{center}
Федеральное государственное автономное образовательное учреждение высшего образования \\
Национальный исследовательский Нижегородский государственный университет им. Н.И. Лобачевского
\end{center}

\begin{center}
Институт информационных технологий, математики и механики
\end{center}

\vspace{4em}

\begin{center}
\textbf{\LargeОтчет по лабораторной работе} \\
\end{center}
\begin{center}
\textbf{\Large«Вычисление многомерных интегралов методом Монте-Карло»} \\
\end{center}

\vspace{4em}

\newbox{\lbox}
\savebox{\lbox}{\hbox{text}}
\newlength{\maxl}
\setlength{\maxl}{\wd\lbox}
\hfill\parbox{7cm}{
\hspace*{5cm}\hspace*{-5cm}\textbf{Выполнил:} \\ студент группы 381708-1 \\ Андронов М. Н.\\
\\
\hspace*{5cm}\hspace*{-5cm}\textbf{Проверил:}\\ доцент кафедры МОСТ, \\ кандидат технических наук \\ Сысоев А. В.
}

\vspace{\fill}

\begin{center} Нижний Новгород \\ 2020 \end{center}

\end{titlepage}

\setcounter{page}{2}

\tableofcontents
\newpage

\section*{Введение}
\addcontentsline{toc}{section}{Введение}
Метод Монте-Карло — общее название группы численных методов, основанных на получении большого числа реализаций стохастического (случайного) процесса, который формируется таким образом, чтобы его вероятностные характеристики совпадали с аналогичными величинами решаемой задачи. Метод Монте-Карло применяется для решения многих задач, как
имеющих, так и не имеющих вероятностную природу (задачи финансовой математики, статистической физики, многомерное интегрирование и др.).
\par Важными достоинствами метода являются: простота реализации, а также, как правило, схема вычислений не зависит от размерности задачи. Сходимость метода – $\frac{1}{\sqrt{N}}$. Для сходимости метода требуется большое количество исходных данных (случайных чисел).
\par Целью данной работы является изучение применения метода Монте-Карло в вычислении многомерных интегралов с использованием различных технологий для выполнения параллельных вычислений.
\newpage

\section*{Постановка задачи}
\addcontentsline{toc}{section}{Постановка задачи}
В рамках данных лабораторных работ ставится задача: реализовать метод Монте-Карло для вычисления многомерных интегралов с использованием различных технологий для выполнения параллельных вычислений.
\par Решение данной задачи разбивается на несколько подзадач:
\begin{itemize}
\item Реализовать последовательный алгоритм вычисления многомерных интегралов методом Монте-Карло.
\item Реализовать параллельный алгоритм, используя OpenMP.
\item Реализовать параллельный алгоритм, используя TBB.
\item Реализовать параллельный алгоритм, используя std::thread.
\item Для каждой из реализаций провести вычислительные эксперименты и проанализировать результаты.
\end{itemize}

\newpage

\section*{Метод решения}
\addcontentsline{toc}{section}{Метод решения}
В задаче численного интрегрирования метод Монте-Карло используется, как правило, для вычисления многомерных интегралов. Например для одномерной функции интеграл можно вычислить методом прямоугольников. Достаточно разбиения на k отрезков и вычисление k значений функции. Если же функция n-мерная (задачи теории струн и т.д.), то по каждой размерности необходимо разбить на k отрезков, следовательно потребуется $k^{n}$ вычислений значения функции. При размерности функции больше 10 задача становится вычислительно трудоемкой.
\par Суть метода Монте-Карло для вычисления интегралов заключается в следующем:
\par Требуется вычислить определённый интеграл $\int_{a}^{b}{f(x)dx}$. Пусть случайная величина $u$ имеет равномерное распределение на интервале $(a;b)$ с плотностью $\varphi {(x)} = \frac{1}{b-a}$. Тогда $f(u)$ также будет случайной величиной причём её математическое ожидание выражается как: \begin{equation}M[f(u)]=\int_{a}^{b}{f(x) \varphi {(x)}dx}\end{equation}. Таким образом искомый интеграл равен: $\int_{a}^{b}{f(x)dx} = (b-a)M[f(u)]$
\par Математическое ожидание случайной величины $f(u)$ можно легко оценить, смоделировав эту случайную величину и посчитав выборочное среднее.
\par Алгоритм будет выглядеть следующим образом:
\begin{enumerate}
\item Выбрать $N$ случайных точек $u_{i}$, равномерно распределенных на интервале $(a;b)$, для каждой точки $u_{i}$ вычислить $f(u_{i})$.
\item По найденным значениям вычислить выборочное среднее $\frac{1}{N}\sum_{i=1}^{N}{f(u_{i})}$
\item Вычислить приближенное значение интервала по формуле: \begin{equation}\int_{a}^{b}{f(x)dx} \approx \frac{b-a}{N}\sum_{i=1}^{N}{f(u_{i})} \end{equation}
\end{enumerate}
\par В многомерном случае алгоритм будет отличаться только тем, что координаты точки будут выбираться из каждой размерности отдельно. В многомерном случае формула будет выглядеть следующим образом: $\int_{D}^{}{f(x_{1}...x_{n})dx} \approx \frac{\prod_{i=1}^{n}{(b_{i}-a_{i})}}{N}\sum_{i=1}^{N}{f(u_{i})}$
\indent D~--- область интегрирования размерности $n$.
\newpage

\section*{Схема распараллеливания}
\addcontentsline{toc}{section}{Схема распараллеливания}
Для сходимости метода Монте-Карло критично качество генератора псевдослучайных чисел (ГПСЧ). Необходимо чтобы каждый поток получал последовательность чисел отличную от последовательностей полученных другими потоками, а также мог сгенерировать ее без взаимодействия с другими потоками. ГПСЧ должен обладать свойством масштабируемости: количество различных генерируемых последовательностей не должно быть привязано к количеству потоков.
\par В качестве генератора псевдослучайных чисел был выбран \emph{mt19937( )}, который при создании будет инициализироваться различными значениями в каждом потоке, что позволит генерировать различные последовательности. 
\par Общая схема работы параллельного алгоритма:
\begin{enumerate}
\item В каждом потоке создать экземпляр генератора псевдослучайных чисел и проинициализировать его уникальным значением.
\item В каждом потоке вычислить сумму значений функции для случайно сгенерированных точек. Количество точек равно порции вычислений для данного потока.
\item В мастер потоке просуммировать значения, полученные на каждом отдельно взятом потоке. Вычислить выборочное среднее и умножить на меру области интегрирования.
\end{enumerate}
\newpage

\section*{Описание программной реализации}
\addcontentsline{toc}{section}{Описание программной реализации}
\subsection*{OpenMP}
\addcontentsline{toc}{subsection}{OpenMP}
\par В OpenMP создание потоков происходит при достижении директивы \verb|#pragma omp| \verb|parallel|. Весь код, который будет выполняться параллельно, должен быть расположен внутри этой директивы.
\par Следуя общей схеме параллельного алгоритма, необходимо создать ГПСЧ и проинициализировать его. Инициализация происходит при помощи класса \verb|std::random_device|.
\par OpenMP предоставляет специальную директиву \verb|for|, которая позволяет распараллеливать циклы, разбиение итераций между потоками происходит в зависимости от параметра \verb|schedule| (по-умолчанию static). Параметр \verb|reduction| позволяет определить список переменных, для которых результат вычислений в отдельных потоках будет собран в мастер-потоке.

\subsection*{TBB}
\addcontentsline{toc}{subsection}{TBB}
Библиотека TBB также предоставляет специальный функцию для выполнения редукции \verb|tbb::parallel_reduce|, которая в качестве обязательных параметров принимает:
 \begin{enumerate}
\item Итерационное пространство \verb|tbb::blocked_range|  ~--- в текущей задаче одномерное итерационное пространоство, которое задает диапазон в виде
полуинтервала [begin, end), где begin = 0, end = N (количество точек)
\item Функтор ~--- класс, реализующий тело цикла через метод \verb|body::operator()|. Для выполнения операции редукции в функтор необходимо добавить реализацию метода \verb|join| который выполняет редукцию. Значение редукции сохраняется в переменной функтора.
\end{enumerate}
Разделение на порции вычислений происходит автоматически благодаря планировщику TBB. ГПСЧ инициализируется в констукторе функтора также за счет использования \verb|std::random_device|

\subsection*{std::thread}
\addcontentsline{toc}{subsection}{std::thread}
В случае std::thread необходимо "вручную" создать объект потока \verb|std::thread|. Для получения количества потоков поддерживаемых системой используется функция \verb|std::thread::hardware_concurrency|. \par Далее вычисляется объем порции вычислений, для каждого потока. Затем создаются потоки, конструктор для создания этого объекта в качестве обязательных параметров принимает: указатель на функцию, которую будет исполнять и параметры это функции.
\par Для ожидания завершения вычислений каждого потока, мастер-потоком последовательно, для каждого потока вызывается метод \verb|join()|. 
\par Редукция результатов, полученных каждым потоком, производится путем использования общей переменной. Для того чтобы избежать "гонки данных" для этой переменной, поток захватывает мьютекс, по окончании вычислений ~--- освобождает.
\newpage

\section*{Подтверждение корректности}
\addcontentsline{toc}{section}{Подтверждение корректности}
Для подтверждения корректности в программе реализован набор тестов, разработанных при помощи библиотеки для модульного тестирования Google C++ Testing Framework. Проверяются случаи вычисления одномерных, двумерных и трехмерных интегралов. Значения полученные при вычислении методом Монте-Карло сравниваются со значениями полученным аналитически.
\par Успешное прохождение всех тестов является подтвержением корректной работы программы. 

\newpage

\section*{Результаты экспериментов}
\addcontentsline{toc}{section}{Результаты экспериментов}
Конфигурация системы:
\begin{itemize}
\item Процессор: Intel(R) Core(TM) i5-9300H CPU @ 2.40GHz
\item Число ядер: 4
\item Оперативная память: 8192 MB (DDR4), 2400 MHz;
\item ОС: Ubuntu 18.04.4 LTS
\end{itemize}

\par Эксперименты проводяться для \verb|123 456 789| случайных точек. 
\par Результаты экспериментов представлены в Таблице 1.

\begin{table}[!h]
\caption{Резултаты вычислительных экспериментов}
\centering
\begin{tabular}{|c|c|c|c|c|c|c|c|}
\hline
\multirow{3}{*}
	{\begin{tabular}[c]{@{}c@{}}Кол-во\\ потоков\end{tabular}} & 
\multirow{2}{*}
	{\begin{tabular}[c]{@{}c@{}}Последовательный\\ алгоритм\end{tabular}} & 
\multicolumn{6}{c|}
	{Параллельный алгоритм}	\\ 
	\cline{3-8} & & 
	\multicolumn{2}{c|}{OpenMP} & 
	\multicolumn{2}{c|}{TBB} & 
	\multicolumn{2}{c|}{std::thread} 
	\\ \cline{2-8}
	& t, с	    & t, с & speedup		& t, с & speedup		& t, с & speedup		\\ \hline
2   & 49.17     & 25.94 & 1.89       	& 24.52 & 2.01        	& 26.01 & 1.89           \\ \hline
4   & 49.17     & 12.61 & 3.89       	& 12.3 & 3.99         	& 13.72  & 3.57          \\ \hline
\end{tabular}
\end{table}

\par По данным, полученным в результате экспериментов, можно сделать вывод, что параллельный алгоритм работает быстрее, чем последовательный приблизительно в n раз, где n ~--- количество потоков. При условии что количество потоков меньше или равно количеству ядер процессора.
\newpage

\section*{Заключение}
\addcontentsline{toc}{section}{Заключение}
В результате выполнения данных лабораторных работ был подробно изучен метод Монте-Карло для вычисление многомерных интегралов.
Реализованы последовательная и параллельные версии алгоритма с использованием технологий: OpenMP, TBB, std::thread.
\par Разработаны тесты подтверждающие корректность работы программы.
\par В заключение проведены эксперименты, подтверждающие эффективность параллельных версий алгоритма. По результатам опытов лучшей технологией для реализации параллельного метода Монте-Карло для вычисление многомерных интегралов оказалась TBB.
\newpage

\begin{thebibliography}{1}
\addcontentsline{toc}{section}{Список литературы}
\bibitem{Sysoev} Сысоев А.В., Мееров И.Б., Свистунов А.Н., Курылев А.Л., Сенин А.В., Шишков А.В., Корняков К.В., Сиднев А.А. «Параллельное программирование в системах с общей памятью. Инструментальная поддержка». Учебно-методические материалы по программе повышения квалификации «Технологии высокопроизводительных вычислений для обеспечения учебного процесса и научных исследований». Нижний Новгород, 2007, 110 с. 
\bibitem{Barkalov} Баркалов К.А. Методы параллельных вычислений. Н. Новгород: Изд-во Нижегородского госуниверситета им. Н.И. Лобачевского, 2011
\bibitem{Wiki1} Wikipedia: the free encyclopedia [Электронный ресурс] // URL: https://en.wikipedia.org/wiki/\verb|Mersenne_Twister| (дата обращения: 20.03.2020)
\end{thebibliography}
\newpage

\section*{Приложение}
\addcontentsline{toc}{section}{Приложение}
\centerline{\bfseries Исходный код.} 


\lstinputlisting[language=C++, caption=Последовательная версия. Заголовочный файл]{../../../../modules/task_1/andronov_m_monte_carlo_integration/monte_carlo_integration.h}
\lstinputlisting[language=C++, caption=Последовательная версия. Cpp файл]{../../../../modules/task_1/andronov_m_monte_carlo_integration/monte_carlo_integration.cpp}
\lstinputlisting[language=C++, caption=Последовательная версия. Тесты]{../../../../modules/task_1/andronov_m_monte_carlo_integration/main.cpp}

\lstinputlisting[language=C++, caption=OpenMP версия. Заголовочный файл]{../../../../modules/task_2/andronov_m_monte_carlo_integration/monte_carlo_integration.h}
\lstinputlisting[language=C++, caption=OpenMP версия. Cpp файл]{../../../../modules/task_2/andronov_m_monte_carlo_integration/monte_carlo_integration.cpp}
\lstinputlisting[language=C++, caption=OpenMP версия. Тесты]{../../../../modules/task_2/andronov_m_monte_carlo_integration/main.cpp}

\lstinputlisting[language=C++, caption=TBB версия. Заголовочный файл]{../../../../modules/task_3/andronov_m_monte_carlo_integration/monte_carlo_integration.h}
\lstinputlisting[language=C++, caption=TBB версия. Cpp файл]{../../../../modules/task_3/andronov_m_monte_carlo_integration/monte_carlo_integration.cpp}
\lstinputlisting[language=C++, caption=TBB версия. Тесты]{../../../../modules/task_3/andronov_m_monte_carlo_integration/main.cpp}

\lstinputlisting[language=C++, caption=std::thread версия. Заголовочный файл]{../../../../modules/task_4/andronov_m_monte_carlo_integration/monte_carlo_integration.h}
\lstinputlisting[language=C++, caption=std::thread версия. Cpp файл]{../../../../modules/task_4/andronov_m_monte_carlo_integration/monte_carlo_integration.cpp}
\lstinputlisting[language=C++, caption=std::thread версия. Тесты]{../../../../modules/task_4/andronov_m_monte_carlo_integration/main.cpp}


\end{document}
