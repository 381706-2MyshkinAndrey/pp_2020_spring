\documentclass{report}

\usepackage[T2A]{fontenc}
\usepackage[utf8]{luainputenc}
\usepackage[english, russian]{babel}
\usepackage[pdftex]{hyperref}
\usepackage[14pt]{extsizes}
\usepackage{listings}
\usepackage{color}
\usepackage{geometry}
\usepackage{enumitem}
\usepackage{multirow}
\usepackage{graphicx}
\usepackage{indentfirst}

\DeclareGraphicsExtensions{.jpg}

\geometry{a4paper,top=2cm,bottom=3cm,left=2cm,right=1.5cm}
\setlength{\parskip}{0.5cm}
\setlist{nolistsep, itemsep=0.3cm,parsep=0pt}

\lstset{language=C++,
		basicstyle=\footnotesize,
		keywordstyle=\color{blue}\ttfamily,
		stringstyle=\color{red}\ttfamily,
		commentstyle=\color{green}\ttfamily,
		morecomment=[l][\color{magenta}]{\#}, 
		tabsize=4,
		breaklines=true,
  		breakatwhitespace=true,
  		title=\lstname,       
}

\makeatletter
\renewcommand\@biblabel[1]{#1.\hfil}
\makeatother

\begin{document}

\begin{titlepage}

\begin{center}
Министерство науки и высшего образования Российской Федерации
\end{center}

\begin{center}
Федеральное государственное автономное образовательное учреждение высшего образования \\
Национальный исследовательский Нижегородский государственный университет им. Н.И. Лобачевского
\end{center}

\begin{center}
Институт информационных технологий, математики и механики
\end{center}

\vspace{4em}

\begin{center}
\textbf{\LargeОтчет по лабораторной работе} \\
\end{center}
\begin{center}
\textbf{\Large«Выделение компонент на бинарном изображении»} \\
\end{center}

\vspace{4em}

\newbox{\lbox}
\savebox{\lbox}{\hbox{text}}
\newlength{\maxl}
\setlength{\maxl}{\wd\lbox}
\hfill\parbox{7cm}{
\hspace*{5cm}\hspace*{-5cm}\textbf{Выполнила:} \\ студент группы 381706-1 \\ Кукушкина К. О.\\
\\
\hspace*{5cm}\hspace*{-5cm}\textbf{Проверил:}\\ доцент кафедры МОСТ, \\ кандидат технических наук \\ Сысоев А. В.
}

\vspace{\fill}

\begin{center} Нижний Новгород \\ 2020 \end{center}

\end{titlepage}

\setcounter{page}{2}

\tableofcontents
\newpage

\section*{Введение}
\addcontentsline{toc}{section}{Введение}

Выделение связных компонент на изображении - одна из важнейших задач компьютерного зрения. После разбиения изображения на связные области можно вести дальнейший  геометрический, логический, топологический и любой другой анализ.
\par Трудоемкость операции достаточно высока, а входные данные в реальных задачах распознавания образов (к примеру) имеют достаточно большой объем. Поэтому целесообразно решать задачу с помощью инструментов параллельного программирования.
\par В этой работе для исследования были выбраны три инструмента:
\begin{itemize}
	\item Open Multi-Processing (OpenMP)
	\item Intel Threading Building Blocks (TBB)
	\item Библиотека поддержки потоков std::threads.
\end{itemize}
\par Цель работы - исследовать эффективность работы алгоритма в разных реализациях.

\newpage

\section*{Постановка задачи}
\addcontentsline{toc}{section}{Постановка задачи}

Для выполнения цели работы поставлены следующие задачи:

\begin{enumerate} 

\item Реализовать последовательный алгоритм выделения компонент на бинарном изображении
\item Реализовать последовательные версии алгоритма с использованием трех инструментов параллельного программирования
\item Написать представительную ситему автоматических тестов для каждой версии алгоритма с помощью Google C++ Testing Framework
\item Провести вычислительнный эксперимент
\item Проанализировать результаты эксперимента, сделать выводы об эффективности технологий применительно к задаче выделения компонент на бинарном изображении.

\end{enumerate} 

\newpage

\section*{Описание алгоритма}
\addcontentsline{toc}{section}{Описание алгоритма}
На вход алгоритма поступает \verb|std::vector<std::size_t>>| (массив беззнаковых целых чисел) размера \verb|h|*\verb|w|, где \verb|h| - высота картинки, \verb|w| - ширина.
\par Последовательная версия задачи реализована следующим образом:
\begin{enumerate}
	\item Сегментация
	\begin{itemize}
		\item Если первый пиксель не пуст, начать новый сегмент
		\item Для всей первой строки: если текущий пиксель не пуст и левый пиксель "окрашен", окрасить текущий тем же цветом; если текущий пиксель не пуст и левый пиксель пуст, начать новый сегмент
		\item Для следующих строк:
		\begin{itemize}
			\item Для первого пикселя каждой строки: если пиксель не пуст и верхний пиксель не пуст, окрасить текущий тем же цветом; если верхний пиксель пуст, начать новый сегмент
			\item Для следующих пикселей:
			\begin{itemize}
				\item Если пиксель пуст, перейти к следующей итерации
				\item Если верхний и левый пиксели не окрашены, начать новый сегмент
				\item Если окрашел левый или верхний пиксель, окрасить текущий тем же цветом
				\item Если окрашены и левый, и верхний пиксели, окрасить текущий в цвет левого.
			\end{itemize}
		\end{itemize}
	\end{itemize}
\item Препроцессинг
\begin{itemize}
	\item Начиная со второй строки: если найден стык сегментов, сохранить для нижнего цвет верхнего
\end{itemize}
\item Перекраска
\begin{itemize}
	\item Перекрасить все пиксели в сохраненные для них цвета
\end{itemize}
\end{enumerate}

\newpage

\section*{Схема распараллеливания}
\addcontentsline{toc}{section}{Схема распараллеливания}
Для всех параллельных версий программ распараллеливаются только секции \verb|Сегментация| и \verb|Перекраска|, т.к. только в них есть возможность разбить циклы на независимые итерации.
\subsection*{OpenMP}
\addcontentsline{toc}{subsection}{OpenMP}
Итерации цикла сегментации динамически распределяются между созданными потоками так, чтобы каждому потоку достался цельный блок из фиксированного количества строк (например, 40*w элементов).
\par Препроцессинг производится последовательно.
\par При перекраске уже неоязательно разбивать массив на блоки определенного размера, распределение происходит по умолчанию.

\subsection*{TBB}
\addcontentsline{toc}{subsection}{TBB}
Создаются два класса-функтора \verb|Segmentation| и \verb|Recolor|. Тела соответствующих циклов перемещены в реализацию методов \verb|operator()|. Размер блоков, обрабатываемых процессами, кратен w.
\par Препроцессинг производится последовательно.

\subsection*{std::threads}
\addcontentsline{toc}{subsection}{std::threads}
Входной массив примерно поровну (остаток строк обрабатывается главным потоком) разделяется между потоками в секциях \verb|Сегментация| и \verb|Перекраска|.
\par Препроцессинг производится последовательно.

\newpage

\section*{Доказательство корректности}
\addcontentsline{toc}{section}{Доказательство корректности}
Для доказательства кореектности реализована система unit-тестов \verb|Bin_image_segmentation| с использованием Google C++ Testing Framework.
\par \verb|can_generate_image| - не возникает исключений при генерации большого изображения;
\par \verb|negative_dim_image| - возникает исключение при генерации изображения отрицательного размера;
\par \verb|image_dim| - размер сгенерированного изображения совпадает с заданными параметрами;
\par \verb|empty_space_small| - не закрашиваются лишние пиксели и не стираются существующие на маленьком изображении;
\par \verb|empty_space_large| - не закрашиваются лишние пиксели и не стираются существующие на большом изображении;
\par \verb|correctness_square| - корректно обрабатывается квадратное изображение;
\par \verb|correctness_rectangle| - корректно обрабатывается прямоугольное изображение.
\par Все тесты проходятся успешно. За 100 запусков на каждой версии программы ни одного проваленного теста зафиксировано не было.
\newpage

\section*{Вычислительный эксперимент}
\addcontentsline{toc}{section}{Вычислительный эксперимент}
Эксперимент проводился на системе со следующими параметрами:
\begin{itemize}
\item Процессор: Intel(R) Core(TM) i5-7200U CPU @ 2.50GHz 2.71GHz
\item ОЗУ: 4,00 Gb
\item ОС: Microsoft Windows 10 Домашняя
\end{itemize}
\par Время работы всех версий программы (усредненное по 10 запускам) на входных данных различного размера занесены в таблицу.
\begin{center}
	\begin{tabular} {| c | c  c  c  c |}
		\hline
		размер & Seq & OpenMP & TBB & std::threads \\
		\hline
		1000*1000 & 0.41s & 0.38s & 0.28s & 0.25s \\
		1200*1200 & 0.72s & 0.61s & 0.50s & 0.53s \\
		1400*1400 & 1.53s & 1.30s & 1.08s & 1.03s \\
		1600*1600 & 2.07s & 1.69s & 1.54s & 1.47s \\
		1800*1800 & 3.54s & 3.13s & 2.61s & 2.56s \\
		2000*2000 & 4.91s & 3.60s & 3.27s & 3.18s \\
		\hline
	\end{tabular}
\end{center}
\par Наименьшую эффективность на данной системе демонстрирует OpenMP-версия программы, наибольшую - версия с использованием std::threads. TBB-версия по эффективности близка к std::threads.
\par У std::threads-версии есть одна проблема - количество потоков фиксировано. То есть, например, при переносе на систему с 4 ядрами эффективность будет не максимальной. В TBB-версии используется пул управляемых потоков, то есть при переносе программа подстроится под новую систему. При том, что время работы TBB-версии близко к std::threads, ее использование кажется более рациональным.

\newpage

\section*{Заключение}
\addcontentsline{toc}{section}{Заключение}
В ходе работы был реализован алгоритм выделения компонент на бинарном изображении в четырех версиях: последовательный и с использованием трех инструментов параллельного программирования.
\par Для подтверждения корректности работы программ разработана система автоматических тестов, учитывающая их основной функционал.
\par Эффективность параллельных версий исследованав ходе вычислительного эксперимента. В рамках конкретной системы, на которой проводилось тестирование, наименее эффективной оказалась OpenMP-версия, наиболее эффективной - версия с использованием std::threads. Однако исходя из соображений переносимости использование TBB будет более оправданным.

\newpage

\begin{thebibliography}{1}
\addcontentsline{toc}{section}{Список литературы}

\bibitem{Gergel} Гергель В.П., Стронгин Р.Г. Основы параллельных вычислений для многопроцессорных вычислительных систем. Учебное пособие – Нижний Новгород: Изд-во ННГУ им. Н.И. Лобачевского, 2003. 184 с. ISBN 5-85746-602-4. 

\bibitem{Habr} Сообщество IT-специалистов Habr. Подсчет числа объектов на бинарном изображении: 
https://habr.com/ru/post/119244/

\end{thebibliography}
\newpage

\section*{Приложение}
\addcontentsline{toc}{section}{Приложение}
\subsection*{Sequential}
\addcontentsline{toc}{subsection}{Sequential}
	\lstinputlisting[language=C++]{../../../../modules/task_1/kukushkina_k_segmentation/bin_image_segmentation.h}
	\lstinputlisting[language=C++]{../../../../modules/task_1/kukushkina_k_segmentation/bin_image_segmentation.cpp}
	\lstinputlisting[language=C++]{../../../../modules/task_1/kukushkina_k_segmentation/main.cpp}
\subsection*{OpenMP}
\addcontentsline{toc}{subsection}{OpenMP}
\lstinputlisting[language=C++]{../../../../modules/task_2/kukushkina_k_segmentation/bin_image_segmentation.h}
\lstinputlisting[language=C++]{../../../../modules/task_2/kukushkina_k_segmentation/bin_image_segmentation.cpp}
\lstinputlisting[language=C++]{../../../../modules/task_2/kukushkina_k_segmentation/main.cpp}
\subsection*{TBB}
\addcontentsline{toc}{subsection}{TBB}
\lstinputlisting[language=C++]{../../../../modules/task_3/kukushkina_k_segmentation_tbb/bin_image_segmentation.h}
\lstinputlisting[language=C++]{../../../../modules/task_3/kukushkina_k_segmentation_tbb/bin_image_segmentation.cpp}
\lstinputlisting[language=C++]{../../../../modules/task_3/kukushkina_k_segmentation_tbb/main.cpp}
\subsection*{std::threads}
\addcontentsline{toc}{subsection}{std::threads}
\lstinputlisting[language=C++]{../../../../modules/task_4/kukushkina_k_segmentation/segmentation.h}
\lstinputlisting[language=C++]{../../../../modules/task_4/kukushkina_k_segmentation/segmentation.cpp}
\lstinputlisting[language=C++]{../../../../modules/task_4/kukushkina_k_segmentation/main.cpp}

\end{document}